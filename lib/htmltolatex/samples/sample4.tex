\documentclass[a4paper,11pt]{article}
\usepackage{ulem}
\usepackage{a4wide}
\usepackage[dvipsnames,svgnames]{xcolor}
\usepackage[pdftex]{graphicx}
\title{Bucket and radix sorting}
\usepackage{hyperref}


\begin{document}

\section{ICS 161: Design and Analysis of Algorithms
  \\
  Lecture notes for January 23, 1996}

\line(1,0){300}




\section{Bucket Sorting}

We've seen various algorithms for sorting in O(n log n) time and a
lower bound showing that O(n log n) is optimal. But the lower bound
only applies to comparison sorting. What if we're allowed to do other
operations than comparisons? The results will have to depend on what
specific data type we want to sort; typical types might be integer,
floating point, or character string.


Let's start with a really simple example: We want to sort n integers
that are all in the range 1..n, no two the same. How quickly can we
determine the sorted order?
\begin{quotation}Answer: O(1), without even computing anything you
  know it has to be 1,2,3,...n-1,n.
\end{quotation}

As a less stupid example, suppose all numbers are still in the range
1..n but some might be duplicates. We can use an array to count how
many copies of each number there are:
\begin{verbatim}
    sort(int n, X[n])
    \{
    int i,j, Y[n+1]
    for (i = 0; i $<$ n; i++) Y[i] = 0;
    for (i = 0; i $<$ n; i++) Y[X[i]]++;
    for (i = 0, j = 0; i $<$ n; i++)
        while(Y[i]-- $>$ 0) X[j++] = i
    \}
\end{verbatim}

The three loops here take O(n) time (we go back to time instead of
counting comparisons since this algorithm doesnt use any comparisons).


One not-completely-obvious point: one of the loops is nested.
Normally when we see a collection of nested loops, the time bound is
the product of the number of iterations of each loop. Why is this
nested loop O(n) rather than O(n\textasciicircum{}2)?
\begin{quotation}Answer: It's possible for the inner loop to execute
  as many as n times, but not on all n iterations of the outer loop.
  The easiest way to see this is to match up the times when we
  increment the array entries with the times when we decrement them.
  Since there are only n increments, there are also only n decrements.
\end{quotation}

\hypertarget{key}{This algorithm is already close to useful. But it is
  less likely that we want to sort numbers exactly, and more likely
  that we want to sort records by some number derived from them. (As
  an example, maybe we have the data from the UCI phone book, and we
  want to sort all the entries by phone number. It's not so useful to
  sort just the phone numbers by themselves; we want to still know
  which name goes with which number.)}

So suppose you have a list of n records each with a key that's a
number from 1 to k (we generalize the problem a little so k is not
necessarily equal to n).

We can solve this by making an array of linked lists. We move each
input record into the list in the appropriate position of the array
then concatenate all the lists together in order.
\begin{verbatim}
    bucket sort(L)
    \{
    list Y[k+1]
    for (i = 0; i $<$= k; i++) Y[i] = empty
    while L nonempty
    \{
        let X = first record in L
        move X to Y[key(X)]
    \}
    for (i = 0; i $<$= k; i++)
    concatenate Y[i] onto end of L
    \}
\end{verbatim}

There are two loops taking O(k) time, and one taking O(n), so the
total time is O(n+k). This is good when k is smaller than n. E.g.
suppose you want to sort 10000 people by birthday; n=10000, k=366, so
time = O(n).


\subsection{Stability}

We say that a sorting algorithm is \textit{stable} if, when two
records have the same key, they stay in their original order. This
property will be important for extending bucket sort to an algorithm
that works well when k is large. But first, which of the algorithms
we've seen is stable?

\begin{itemize}
\item Bucket sort? Yes. We add items to the lists Y[i] in order, and
  concatenating them preserves that order.
\item Heap sort? No. The act of placing objects into a heap (and
  heapifying them) destroys any initial ordering they might have.
\item Merge sort? Maybe. It depends on how we divide lists into two,
  and on how we merge them. For instance if we divide by choosing
  every other element to go into each list, it is unlikely to be
  stable. If we divide by splitting a list at its midpoint, and break
  ties when merging in favor of the first list, then the algorithm can
  be stable.
\item Quick sort? Again, maybe. It depends on how you do the partition
  step.
\end{itemize}

Any comparison sorting algorithm can be made stable by modifying the
comparisons to break ties according to the original positions of the
objects, but only some algorithms are automatically stable.


\subsection{Radix sort}

What to do when k is large? Think about the decimal representation of
a number

\begin{verbatim}
    x = a + 10 b + 100 c + 1000 d + ...
\end{verbatim}

where a,b,c etc all in range 0..9. These digits are easily small
enough to do bucket sort.

\begin{verbatim}
    radix sort(L):
    \{
    bucket sort by a
    bucket sort by b
    bucket sort by c
    ...
    \}
\end{verbatim}

or more simply

\begin{verbatim}
    radix sort(L):
    \{
    while (some key is nonzero)
    \{
        bucket sort(keys mod 10)
        keys = keys / 10
    \}
    \}
\end{verbatim}

The only possibly strange part: Why do we do the sort least important
digit first? For that matter, why do we do more than one bucket sort,
since the last one is the one that puts everything into place?

\begin{quotation}Answer: If we're trying to sort things by hand we
  tend to do something different: first do a bucket sort, then
  recursively sort the values sharing a common first digit. This
  works, but is less efficient since it splits the problem up into
  many subproblems. By contrast, radix sorting never splits up the
  list; it just applies bucket sorting several times to the same list.


  In radix sorting, the last pass of bucket sorting is the one with
  the most effect on the overall order. So we want it to be the one
  using the most important digits. The previous bucket sorting passes
  are used only to take care of the case in which two items have the
  same key (mod 10) on the last pass.
\end{quotation}

Correctness:
\begin{quotation}We prove that the algorithm is correct by induction.
  The induction hypothesis is that after i steps, the numbers are
  sorted by key modulo 10\textasciicircum{}i. Certainly after no
  steps, all numbers are the same modulo 1, and are therefore sorted
  by that value, so the base case is true. Inductively, step i+1 sorts
  by key / 10\textasciicircum{}i.  If two numbers have the same value
  of key/10\textasciicircum{}i, the stability property of bucket
  sorting leaves them sorted by lower order digits; and if they don't
  have the same value, the bucket sort on step i+1 puts them in the
  right order, so in either case the induction hypothesis holds. For i
  sufficiently large, taking the keys mod 10\textasciicircum{}i
  doesn't change them, at which point the list is sorted.
\end{quotation}

Analysis:
\begin{quotation}The algorithm takes O(n) time per bucket sort.
  \\
  There are log\_10 k = O(log n) bucket sorts.
  \\
  So the total time is O(n log k).
\end{quotation}

Is this ever the best algorithm to use?
\begin{quotation}Answer: No. If k is smaller than n, this takes O(n
  log k) while bucket sort takes only O(n). And if k is larger than n,
  the O(n log k) taken by this method is worse than the O(n log n)
  taken by comparison sorting.
\end{quotation}

How can we make it better?
\begin{quotation}Answer: don't use decimal notation. Multiplication
  and division is expensive, so it's better to use a base that's a
  power of 2 (this saves a constant factor but is an easy
  optimization).  More importantly, 10 is too small; we can increase
  the base up to as large as O(n) without increasing the bucket sort
  times significantly. If we use base n notation (or base some power
  of 2 close to n) we get time bounds of the form

\begin{verbatim}
    O(n (1 + (log k)/(log n)))
\end{verbatim}
\end{quotation}

Example: sorting a million 32-bit numbers. If we use
2\textasciicircum{}16 (roughly 64000) as our base (i.e. if we use this
number in place of 10 in the radix sorting pseudocode above) then the
two O(k) loops are a minuscule fraction of the total time per bucket
sort, and we only need two bucket sorting passes to solve the problem.


With some more complicated methods one can make integer sorting
algorithms having bounds O(n loglog k) or O(n log n / log log n) but
these are of less practical interest, and are only good when k is
enormously greater than n.

\subsection{Sorting floating point numbers}

Floating point represents numbers roughly in the form x = a *
2\textasciicircum{}b.  If all numbers had same value of b, we could
just bucket sort by a.  Different b's complicate the picture but
basically b is more important than a so sort by a first (radix sort)
then one more bucket sort by b.


\subsection{Sorting character strings}

Essentially, alphabetical order no different than a base-26
representation of numbers and sorting by ascii character value is no
different than a base-256 representation. As usual, the first
character is the most important and therefore the one you want to save
for last -- we can just work backwards through the strings doing
bucket sorts by the character values at each position. But there are
some details involved in applying this idea to get a quick algorithm.


What if strings have different lengths? There doesn't always exist a
character in position i. But we can test i against the string's
length, and use character value 0 if the string is too short since
e.g. "a" should come before "aardvark". One way to think about this is
that we just "pad" the strings out with null characters to make them
all the same length.

However, this padding idea can significantly increase the size of the
strings (or in other words it makes a sorting algorithm based on it
too slow). If you have n strings, most very short, but one of length
n, the time will be O(n\textasciicircum{}2) even though the input
length is O(n), so this idea of using radix sort for strings would end
up no better than a bad comparison sort algorithm.

The problem is that when we sort the strings by their i'th character
values, we're wasting a lot of work sorting the strings shorter than
that. One possible solution would be to, in each bucket sort, only
sort the strings longer than that length. How do we find this list of
long strings? We can do it by another call to bucket sorting! Just
sort the strings by their lengths.

There still remains a more subtle problem with this approach.  Recall
that bucket sort is efficient only when n$<$k. In early passes of the
radix sort algorithm, we'll only be sorting really long strings, so
there may be very few of them, and we won't have n$<$k, so those
bucket sorts may not be efficient

Solution: We'll use the same array of k buckets for each radix sort
pass, so putting things into buckets (distribute) is still fast.
Concatenating nonempty buckets (coalesce) is also still fast.  The
only slowdown comes from finding the nonempty buckets. So we make a
list of which buckets will be nonempty using bucket sorting in yet a
third way! We sort pairs (i,string[i]) by two bucket sorts. The
buckets of the second bucket sort give, for each i, a sorted list of
the characters in position i of the strings, which tells us which
buckets will be nonempty in the radix sort part of the algorithm.

This is getting pretty complicated and confusing, but it's not too bad
once we express it in pseudocode:
\begin{verbatim}
    string sort(L)
    \{
    make list of pairs (i,str[i])
    bucket sort pairs by str[i]
    bucket sort pairs by i (giving lists chars[i])
    bucket sort strings by length
    i = max length
    L = empty
    while (i $>$ 0)
    \{
        concatenate strings of length = i before start of L
        distribute into buckets by chars in position i
        coalesce by concatenating buckets in chars[i]
        i--
    \}
    concatenate list of empty strings at start of L
    return L
    \}
\end{verbatim}

The time for the first three bucket sorts is O(N+k). Each remaining
pass takes time O(n\_i) where n\_i is the number of strings having a
character in position i, so the total time for the while loop is O(N)
again. Overall, the algorithm's total time is O(N + k), where N is the
total length of all strings and k is number of character values.


\line(1,0){300}


\href{http://www.ics.uci.edu/%7Eeppstein/161/}{ICS 161} -- \href{http://www.ics.uci.edu/}{Dept. 
Information \& Computer Science} -- \href{http://www.uci.edu/}{UC Irvine}
\\\scriptsize Last update: 
02 May 2000, 20:24:49 PDT \normalsize

\end{document}
